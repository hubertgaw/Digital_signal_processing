\documentclass[12pt]{article}
\usepackage[T1]{fontenc}
\usepackage[T1]{polski}
\usepackage[utf8]{inputenc}
\newcommand{\BibTeX}{{\sc Bib}\TeX} 
\usepackage{graphicx}
\usepackage{amsfonts}
\usepackage{amsmath}
\usepackage{latexsym}
\usepackage{float}

\setlength{\textheight}{21cm}

\title{{\bf Zadanie nr 1 - Generacja sygnału i szumu}\linebreak
Cyfrowe Przetwarzanie Sygnałów}
\author{Dawid Jakubik, 224307 \and Hubert Gawłowski, 224298}
\date{22.04.2021}

\begin{document}
\clearpage\maketitle
\thispagestyle{empty}
\newpage
\setcounter{page}{1}
\section{Cel zadania}

Celem zadania było zapoznanie się z procesem konwersji A/C oraz C/A sygnałów jednocześnie poznając kilka metod na przeprowadzenie obu procesów jak i zaiplementowanie ich w aplikacji. 

\section{Wstęp teoretyczny}

Sygnały, które podlegaja kwantyzacji oraz rekonstrukcji generowane są na podstawie wzorów znajdujących się w instrukcji do zadania nr 1 \cite{instrukcja1}. W instrukcji \cite{instrukcja2} znajdują się wzory oraz opisy metod konwersji A/C oraz C/A użyte w celu kwantyzacji, rekonstrukcji oraz do obliczenia błędu średniokwadratowego, stosunku sygnał-szum, szczytowego stosunku sygnał-szum oraz maksymalnej różnicy.


\section{Eksperymenty i wyniki}

%%%%%%%%%%%%%%%%%%%%%%%%%%%%%%%%%%%%%%%%%%%%%%%%%%%%%%%%%%%%%%%%%%%%%%%%%%%%%%%%%%%%%%%%%%%%%%%%%%%%%%%%%%%%%%%%%
% PODROZDZIA£ PT. EKSPERYMENT NR 1 
%%%%%%%%%%%%%%%%%%%%%%%%%%%%%%%%%%%%%%%%%%%%%%%%%%%%%%%%%%%%%%%%%%%%%%%%%%%%%%%%%%%%%%%%%%%%%%%%%%%%%%%%%%%%%%%%%
W każdym z eksperymantów jako sygnał ciągły posłużył nam sygnał sinusoidalny o następujących parametrach: 
\begin{itemize}
	\item Amplituda: 4
	\item Czas początkowy: 0s
	\item Czas trwania sygnału: 2s
	\item Okres: 1s
\end{itemize}
Wygląda on następująco:
\begin{figure}[H]
    \centering
    %\includegraphics{cps_kwantyzacja_z_obcieciem.jpg}
	\includegraphics[width=\linewidth]{sygnał_sinusoidalny.jpg}
    \caption{Sygnał sinusoidalny wykorzystywany w eksperymentach}
    \label{wykres dla eksperymentu 1}
\end{figure}


\subsection{Eksperyment nr 1 Kwantyzacja równomierna z obcięciem}

Celem eksperymentu było przeprowadzenie kwantyzacji sygnału ciągłego z obcięciem.

\subsubsection{Założenia}

\begin{itemize}
    \item Częstotliwość próbkowania: 15 Hz
    \item Liczba poziomów kwantyzacji: 5
\end{itemize}
\subsubsection{Przebieg}
Skwantyzowana postać sygnału prezentuje się następująco:
\begin{figure}[H]
    \centering
    %\includegraphics{cps_kwantyzacja_z_obcieciem.jpg}
	\includegraphics[width=\linewidth]{kwantyzacja_z_obcieciem.jpg}
    \caption{Kwantyzacja równomierna z obcięciem sygnału sinusoidalnego}
    \label{wykres dla eksperymentu 1.1}
\end{figure}

\subsubsection{Rezultat}
Obliczone miary dla procesu kwantyzacji:
\begin{figure}[H]
    \centering
    %\includegraphics{cps_kwantyzacja_z_obcieciem_miary.jpg}
	\includegraphics[width=\linewidth]{wyniki_kwantyzacja_z_obcieciem.jpg}
    \caption{Miary obliczone dla kwantyzacji równomiernej z obcięciem sygnału sinusoidalnego}
    \label{Wartości dla eksperymentu 1}
\end{figure}


%%%%%%%%%%%%%%%%%%%%%%%%%%%%%%%%%%%%%%%%%%%%%%%%%%%%%%%%%%%%%%%%%%%%%%%%%%%%%%%%%%%%%%%%%%%%%%%%%%%%%%%%%%%%%%%%%

%%%%%%%%%%%%%%%%%%%%%%%%%%%%%%%%%%%%%%%%%%%%%%%%%%%%%%%%%%%%%%%%%%%%%%%%%%%%%%%%%%%%%%%%%%%%%%%%%%%%%%%%%%%%%%%%%

\newpage
\subsection{Eksperyment nr 2 : Kwantyzacja równomierna z zaokrąglaniem}

Celem eksperymentu było przeprowadzenie kwantyzacji sygnału ciągłego z zaokrągleniem.

\subsubsection{Założenia}

\begin{itemize}
	\item Częstotliwość próbkowania: 20 Hz
	\item Liczba poziomów kwantyzacji: 8
\end{itemize}
\subsubsection{Przebieg}
Skwantyzowana postać sygnału prezentuje sie następująco:
\begin{figure}[H]
	\centering
	%\includegraphics{cps_kwantyzacja_z_zaokrągleniem.jpg}
	\includegraphics[width=\linewidth]{sygnal_kwantyzacja_z_zaokragleniem.jpg}
	\caption{Kwantyzacja równomierna z zaokrągleniem sygnału sinusoidalnego}
	\label{wykres dla eksperymentu 2}
\end{figure}

\subsubsection{Rezultat}
Obliczone miary dla procesu kwantyzacji:
\begin{figure}[H]
	\centering
	%\includegraphics{cps_kwantyzacja_z_zaokrągleniem_miary.jpg}
	\includegraphics[width=\linewidth]{wyniki_kwantyzacja_z_zaokragleniem.jpg}
	\caption{Miary obliczone dla kwantyzacji równomiernej z zaokrągleniem sygnału sinusoidalnego}
	\label{Wartości dla eksperymentu 2}
\end{figure}


%%%%%%%%%%%%%%%%%%%%%%%%%%%%%%%%%%%%%%%%%%%%%%%%%%%%%%%%%%%%%%%%%%%%%%%%%%%%%%%%%%%%%%%%%%%%%%%%%%%%%%%%%%%%%%%%%

%%%%%%%%%%%%%%%%%%%%%%%%%%%%%%%%%%%%%%%%%%%%%%%%%%%%%%%%%%%%%%%%%%%%%%%%%%%%%%%%%%%%%%%%%%%%%%%%%%%%%%%%%%%%%%%%%
\newpage
\subsection{Eksperyment nr 3: Ekstrapolacja zerowego rzędu}


\subsubsection{Założenia}
Eksperyment nr 3 polegał na ekstrapolacji zerowego rzędu spróbkowanego sygnału ciągłego.
\subsubsection{Przebieg}
Wygenerowany sygnał prezentuje się następująco:
\begin{figure}[H]
    \centering
    %\includegraphics{cps_ekstrapolacja_0.jpg}
	\includegraphics[width=\linewidth]{sygnal_rekonstrukcja_zero.jpg}
    \caption{Wykres Ekstrapolacji zerowego rzędu sygnału sinusoidalnego}
    \label{wykres dla eksperymentu 3}
\end{figure}



\subsubsection{Rezultat}
Obliczone parametry dla otrzymanego w rezuntacie sygnału ciągłego:
\begin{figure}[H]
    \centering
    %\includegraphics{cps_ekstrapolacja_0_miary.jpg}
	\includegraphics[width=\linewidth]{wyniki_rekonstrukcja_zero.jpg}
    \caption{Oblicznone miary dla Ekstrapolacji zerowego rzędu sygnału sinusoidalnego}
    \label{wartości dla eksperymentu 3}
\end{figure}


%%%%%%%%%%%%%%%%%%%%%%%%%%%%%%%%%%%%%%%%%%%%%%%%%%%%%%%%%%%%%%%%%%%%%%%%%%%%%%%%%%%%%%%%%%%%%%%%%%%%%%%%%%%%%%%%%

%%%%%%%%%%%%%%%%%%%%%%%%%%%%%%%%%%%%%%%%%%%%%%%%%%%%%%%%%%%%%%%%%%%%%%%%%%%%%%%%%%%%%%%%%%%%%%%%%%%%%%%%%%%%%%%%%
\newpage
\subsection{Eksperyment nr 4: Interpolacja pierwszego rzędu}

\subsubsection{Założenia}
Eksperyment nr 4 polegał na Interpolacji pierwszego rzędu spróbkowanego sygnału ciągłego.
\subsubsection{Przebieg}
Wygenerowany sygnał prezentuje się następująco:
\begin{figure}[H]
	\centering
	%\includegraphics{cps_interpolacja_1.jpg}
	\includegraphics[width=\linewidth]{sygnal_interpolacja_pierwszy.jpg}
	\caption{Wykres Interpolacji pierwszego rzędu sygnału sinusoidalnego}
	\label{wykres dla eksperymentu 4}
\end{figure}



\subsubsection{Rezultat}
Obliczone parametry dla otrzymanego w rezultacie sygnału ciągłego:
\begin{figure}[H]
	\centering
	%\includegraphics{cps_interpolacja_1_miary.jpg}
	\includegraphics[width=\linewidth]{wyniki_interpolacja_pierwszy.jpg}
	\caption{Oblicznone miary dla Interpolacja pierwszego rzędu sygnału sinusoidalnego}
	\label{wartości dla eksperymentu 4}
\end{figure}

%%%%%%%%%%%%%%%%%%%%%%%%%%%%%%%%%%%%%%%%%%%%%%%%%%%%%%%%%%%%%%%%%%%%%%%%%%%%%%%%%%%%%%%%%%%%%%%%%%%%%%%%%%%%%%%%%

%%%%%%%%%%%%%%%%%%%%%%%%%%%%%%%%%%%%%%%%%%%%%%%%%%%%%%%%%%%%%%%%%%%%%%%%%%%%%%%%%%%%%%%%%%%%%%%%%%%%%%%%%%%%%%%%%
\newpage
\subsection{Eksperyment nr 5: Rekonstrukcja w oparciu o funkcję sinc }
\subsubsection{Założenia}
Eksperyment nr 3 polegał na rekonstrukcji w oparciu o funkcję sinc spróbkowanego sygnału ciągłego.
\subsubsection{Przebieg}
Wygenerowany sygnał prezentuje się następująco:
\begin{figure}[H]
	\centering
	%\includegraphics{cps_rekonstrukcja_sinc.jpg}
	\includegraphics[width=\linewidth]{sygnal_sinc.jpg}
	\caption{Wykres rekonstrukcji w oparciu o funkcję sinc sygnału sinusoidalnego}
	\label{wykres dla eksperymentu 5}
\end{figure}



\subsubsection{Rezultat}
Obliczone parametry dla otrzymanego w rezuntacie sygnału ciągłego:
\begin{figure}[H]
	\centering
	%\includegraphics{cps_rekonstrukcja_sincmiary.jpg}
	\includegraphics[width=\linewidth]{wyniki_sinc.jpg}
	\caption{Oblicznone miary dla rekonstrukcji w oparciu o funkcję sinc sygnału sinusoidalnego}
	\label{wartości dla eksperymentu 5}
\end{figure}
%%%%%%%%%%%%%%%%%%%%%%%%%%%%%%%%%%%%%%%%%%%%%%%%%%%%%%%%%%%%%%%%%%%%%%%%%%%%%%%%%%%%%%%%%%%%%%%%%%%%%%%%%%%%%%%%%

%%%%%%%%%%%%%%%%%%%%%%%%%%%%%%%%%%%%%%%%%%%%%%%%%%%%%%%%%%%%%%%%%%%%%%%%%%%%%%%%%%%%%%%%%%%%%%%%%%%%%%%%%%%%%%%%%
\newpage
\subsection{Eksperyment nr 6: Zjawisko aliasingu }
\subsubsection{Założenia}
Eksperyment nr 6 polegał na rekonstrukcji w oparciu o funkcję sinc spróbkowanego sygnału ciągłego z tak dobranymi parametrami by uzyskac zjawisko aliasingu.
\subsubsection{Przebieg}
Wygenerowane sygnały sinusoidalne o częstotliwościach 100Hz, 220Hz oraz 440Hz  zostały spróbkowane kolejno z częstotliwościami  1 000Hz, 44 100Hz oraz22 050Hz a następnie sygnały zostały zrekonstruowane za pomocą finkcji sinc.

\subsubsection{Rezultat}
\begin{figure}[H]
	\centering
	%\includegraphics{cps_aliasing_1.jpg}
	\includegraphics[width=\linewidth]{tmp.jpg}
	\caption{Wykres rekonstrukcji w oparciu o funkcję sinc sygnału sinusoidalnego z wystąpieniem aliasingu dla częstotliwości $f_0$=100Hz oraz $f_s$=1 000Hz}
	\label{wykres dla eksperymentu 6}
\end{figure}
Obliczone parametry dla otrzymanego w rezuntacie sygnału ciągłego:
\begin{figure}[H]
	\centering
	%\includegraphics{cps_aliasing_1_miary.jpg}
	\includegraphics[width=\linewidth]{tmp.jpg}
	\caption{Oblicznone miary dla rekonstrukcji w oparciu o funkcję sinc sygnału sinusoidalnego z wystąpieniem aliasingu dla częstotliwości $f_0$=100Hz oraz $f_s$=1 000Hz} 
	\label{wartości dla eksperymentu 6}
\end{figure}
\begin{figure}[H]
	\centering
	%\includegraphics{cps_aliasing_2.jpg}
	\includegraphics[width=\linewidth]{tmp.jpg}
	\caption{Wykres rekonstrukcji w oparciu o funkcję sinc sygnału sinusoidalnego z wystąpieniem aliasingu dla częstotliwości $f_0$=220Hz oraz $f_s$=44 100Hz}
	\label{wykres dla eksperymentu 6.2}
\end{figure}
Obliczone parametry dla otrzymanego w rezuntacie sygnału ciągłego:
\begin{figure}[H]
	\centering
	%\includegraphics{cps_aliasing_2_miary.jpg}
	\includegraphics[width=\linewidth]{tmp.jpg}
	\caption{Oblicznone miary dla rekonstrukcji w oparciu o funkcję sinc sygnału sinusoidalnego z wystąpieniem aliasingu dla częstotliwości $f_0$=220Hz oraz $f_s$=44 100Hz} 
	\label{wartości dla eksperymentu 6.2}
\end{figure}
\begin{figure}[H]
	\centering
	%\includegraphics{cps_aliasing_3.jpg}
	\includegraphics[width=\linewidth]{tmp.jpg}
	\caption{Wykres rekonstrukcji w oparciu o funkcję sinc sygnału sinusoidalnego z wystąpieniem aliasingu dla częstotliwości $f_0$=440Hz oraz $f_s$=22 050Hz}
	\label{wykres dla eksperymentu 6.3}
\end{figure}
Obliczone parametry dla otrzymanego w rezuntacie sygnału ciągłego:
\begin{figure}[H]
	\centering
	%\includegraphics{cps_aliasing_3_miary.jpg}
	\includegraphics[width=\linewidth]{tmp.jpg}
	\caption{Oblicznone miary dla rekonstrukcji w oparciu o funkcję sinc sygnału sinusoidalnego z wystąpieniem aliasingu dla częstotliwości $f_0$=440Hz oraz $f_s$=22 050Hz}
	\label{wartości dla eksperymentu 6.3}
\end{figure}
%%%%%%%%%%%%%%%%%%%%%%%%%%%%%%%%%%%%%%%%%%%%%%%%%%%%%%%%%%%%%%%%%%%%%%%%%%%%%%%%%%%%%%%%%%%%%%%%%%%%%%%%%%%%%%%%%

%%%%%%%%%%%%%%%%%%%%%%%%%%%%%%%%%%%%%%%%%%%%%%%%%%%%%%%%%%%%%%%%%%%%%%%%%%%%%%%%%%%%%%%%%%%%%%%%%%%%%%%%%%%%%%%%%
\section{Wnioski}
\begin{itemize}
    \item Każdy sygnał ciągły można zdyskretyzować poprzez próbkowanie a następnie te próbki skwantyzować by ograniczyć dokładność zapisu oraz ilość pamięci potrzebnej do zapisu sygnału zdyskrekretyzowanego.
    \item Program poprawnie kwantyzuje oraz rekonstruuje sygnały.
    \item Program pozwala obliczenie 4 miar dla każdego skwantyzowanego sygnału.
    \item Najbardziej przypominający kształtem sygnał zrekonstruowany powstaje po rekonstrukcji z wykorzystaniem funkcji sinc.
    \item Kwantyzacja pozwala w efektywny sposób przedstawić sygnał w postaci cyfrowej
    \item Im większa częstotliwość próbkowania, tym jakość rekonstrukcji jest lepsza
    \item Im więcej więcej poziomów kwantyzacji, tym błąd kwantyzacji jest mniejszy.
    \item Przy odpowiednich warunkach może wystąpić zjawisko aliasingu.
\end{itemize}
 

%%%%%%%%%%%%%%%%%%%%%%%%%%%%%%%%%%%%%%%%%%%%%%%%%%%%%%%%%%%%%%%%%%%%%%%%%%%%%%%%%%%%%%%%%%%%%%%%%%%%%%%%%%%%%%%%%
% BIBLIOGRAFIA
%%%%%%%%%%%%%%%%%%%%%%%%%%%%%%%%%%%%%%%%%%%%%%%%%%%%%%%%%%%%%%%%%%%%%%%%%%%%%%%%%%%%%%%%%%%%%%%%%%%%%%%%%%%%%%%%%
\begin{thebibliography}{}
\bibitem{instrukcja1} Instrukcja do zadania 1 na stronie przedmiotu. [przeglądany 28.04.2021], Dostępny w: https://ftims.edu.p.lodz.pl/file.php/154/zadanie120101011.pdf
\bibitem{instrukcja2} Instrukcja do zadania 2 na stronie przedmiotu. [przeglądany 28.04.2021], Dostępny w: {https://ftims.edu.p.lodz.pl/pluginfile.php/13449/modresource/content/0/zadanie2.pdf}

\end{thebibliography}



\end{document}
