\documentclass[12pt]{article}
\usepackage[T1]{fontenc}
\usepackage[T1]{polski}
\usepackage[utf8]{inputenc}
\newcommand{\BibTeX}{{\sc Bib}\TeX} 
\usepackage{graphicx}
\usepackage{amsfonts}
\usepackage{amsmath}
\usepackage{latexsym}
\usepackage{float}

\setlength{\textheight}{21cm}

\title{{\bf Zadanie nr 4 }\linebreak
Cyfrowe Przetwarzanie Sygnałów}
\author{Dawid Jakubik, 224307 \and Hubert Gawłowski, 224298}
\date{17.06.2021}

\begin{document}
\clearpage\maketitle
\thispagestyle{empty}
\newpage
\setcounter{page}{1}
\section{Cel zadania}

Celem zadania było zapoznanie się z operacjami transformacji sygnałów dyskretnych przy użyciu wybranych metod oraz zaimplementowanie tychże metod w celu wygenerowania odpowiednich transformat.

\section{Wstęp teoretyczny}
Podczas pracy nad zadaniem korzystaliśmy z teorii zawartej w instrukcji na platformie Wikamp \cite{instrukcja}. Znajdują się w niej wszystkie potrzebne wzory dotyczące interesujących nas transformacji.
Oczywiście, zgodnie z instrukcją zapewniliśmy dwa tryby wyświetlania wykresu:
\begin{itemize}
    \item (W1) - górny wykres prezentuje część rzeczywistą amplitudy w funkcji częstotliwości, a wykres dolny część urojoną.
    \item (W2) - górny wykres prezentuje moduł liczby zespolonej, a dolny argument liczby w funkcji częstotliwości.
\end{itemize}
W ramach zadania zaimplementowaliśmy wymienione w instrukcji transformacje sygnałów. Z racji numerów indeksów przydzielony nam został Zestaw 2, a więc wariant transformacji falkowej (DB4)


\section{Eksperymenty i wyniki}

%%%%%%%%%%%%%%%%%%%%%%%%%%%%%%%%%%%%%%%%%%%%%%%%%%%%%%%%%%%%%%%%%%%%%%%%%%%%%%%%%%%%%%%%%%%%%%%%%%%%%%%%%%%%%%%%%
% PODROZDZIA£ PT. EKSPERYMENT NR 1 
%%%%%%%%%%%%%%%%%%%%%%%%%%%%%%%%%%%%%%%%%%%%%%%%%%%%%%%%%%%%%%%%%%%%%%%%%%%%%%%%%%%%%%%%%%%%%%%%%%%%%%%%%%%%%%%%%

\subsection{Wykresy wejściowe}

Na początek wygenerowaliśmy wykresy o następujących wzorach, na których będą przeprowadzane transformacje:
\begin{itemize}
    \item (S1):
    \begin{equation}
        S(t) = 2 \sin (\frac{2 \pi}{2}t + \frac{\pi}{2}) + 5 \sin (\frac{2 \pi}{0.5}t + \frac{\pi}{2})f_{pr} = 16
    \end{equation}
    \item (S2):
    \begin{equation}
        S(t) = 2 \sin (\frac{2 \pi}{2}t) + \sin (\frac{2 \pi}{1}t)+5 \sin (\frac{2\pi}{0.5}t) f_{pr} = 16
    \end{equation}
    \item (S3):
    \begin{equation}
        S(t) = 5 \sin (\frac{2 \pi}{2}t) + \sin (\frac{2 \pi}{0.25}t)f_{pr} = 16
    \end{equation}
\end{itemize}

% zeby wygenerować podałem dla sinusoidalnych wartosci:
% Pierwszy: A:2, ts: 1.39, czas trwania:4, T:2
% Drugi: A:5, ts:1.39, czas trwania:4, T:0.5
Wygenerowane wykresy przedstawiają się następująco:
\begin{figure}[H]
	\centering
	\includegraphics[width=\linewidth]{S1.png}
	\caption{Wykres sygnału S1}
	\label{S1_sygnal}
\end{figure}

% zeby wygenerować podałem dla sinusoidalnych wartosci:
% Pierwszy: A:2, ts: 0, czas trwania:4, T:2
% Drugi: A:1, ts:0, czas trwania:4, T:1
% i potem klkinałem "Dodaj do wygenrowanego; wtedy dodaje do shardkodowanego sygnału
\begin{figure}[H]
	\centering
	\includegraphics[width=\linewidth]{S2.png}
	\caption{Wykres sygnału S2}
	\label{S2_sygnal}
\end{figure}

% zeby wygenerować podałem dla sinusoidalnych wartosci:
% Pierwszy: A:5, ts: 0, czas trwania:4, T:2
% Drugi: A:1, ts:0, czas trwania:4, T:0.25

\begin{figure}[H]
	\centering
	\includegraphics[width=\linewidth]{S3.png}
	\caption{Wykres sygnału S3}
	\label{S3_sygnal}
\end{figure}


\subsection {Dyskretna transformata Fouriera}
Każda z poniższych grafik przedstawia wykresy otrzymane z punktów uzyskanych z transformacji. Każda z nich w zależności od typu pary wykresów (W1 – górny wykres prezentuje część rzeczywistą amplitudy w funkcji
częstotliwości, a wykres dolny część urojoną, W2 – górny wykres prezentuje moduł liczby zespolonej, a dolny argument liczby w funkcji częstotliwości.) przedstawia różne informacje. W opisach każdej z grafik zawarto rodzaj sygnału, czas transformacji oraz typ wykresów.

\begin{figure}[H]
	\centering
	\includegraphics[width=.8\linewidth]{DFT-S1-W1}
	\caption{Wykresy w formie W1 DFT dla sygnału S1. Czas obliczania: 1,118799s}
	\label{S1_sygnal}
\end{figure}
\begin{figure}[H]
	\centering
	\includegraphics[width=.8\linewidth]{DFT-S1-W2}
	\caption{Wykresy w formie W2 DFT dla sygnału S1. Czas obliczania: 1,123200s}
	\label{S3_sygnal}
\end{figure}
\begin{figure}[H]
	\centering
	\includegraphics[width=.8\linewidth]{FFT-S1-W1}
	\caption{Wykresy w formie W1 FFT dla sygnału S1. Czas obliczania: 0,233199s}
	\label{S3_sygnal}
\end{figure}
\begin{figure}[H]
	\centering
	\includegraphics[width=.8\linewidth]{FFT-S1-W2}
	\caption{Wykresy w formie W2 FFT dla sygnału S1. Czas obliczania: 0,102800s}
	\label{S3_sygnal}
\end{figure}


\begin{figure}[H]
	\centering
	\includegraphics[width=.8\linewidth]{DFT-S2-W1}
	\caption{Wykresy w formie W1 DFT dla sygnału S2. Czas obliczania: 1,128499s}
	\label{S3_sygnal}
\end{figure}
\begin{figure}[H]
	\centering
	\includegraphics[width=.8\linewidth]{DFT-S2-W2}
	\caption{Wykresy w formie W2 DFT dla sygnału S2. Czas obliczania: 2,854500s}
	\label{S3_sygnal}
\end{figure}
\begin{figure}[H]
	\centering
	\includegraphics[width=.8\linewidth]{FFT-S2-W1}
	\caption{Wykresy w formie W1 FFT dla sygnału S2. Czas obliczania: 0,079100s}
	\label{S3_sygnal}
\end{figure}
\begin{figure}[H]
	\centering
	\includegraphics[width=.8\linewidth]{FFT-S2-W2}
	\caption{ Wykresy w formie W2 FFT dla sygnału S2. Czas obliczania: 0,097300s}
	\label{S3_sygnal}
\end{figure}


\begin{figure}[H]
	\centering
	\includegraphics[width=.8\linewidth]{DFT-S3-W1}
	\caption{Wykresy w formie W1 DFT dla sygnału S3. Czas obliczania: 1,250000s}
	\label{S3_sygnal}
\end{figure}
\begin{figure}[H]
	\centering
	\includegraphics[width=.8\linewidth]{DFT-S3-W2}
	\caption{Wykresy w formie W2 DFT dla sygnału S3. Czas obliczania: 1,404800s}
	\label{S3_sygnal}
\end{figure}
\begin{figure}[H]
	\centering
	\includegraphics[width=.8\linewidth]{FFT-S3-W1}
	\caption{Wykresy w formie W1 FFT dla sygnału S3. Czas obliczania: 0,086600s}
	\label{S3_sygnal}
\end{figure}
\begin{figure}[H]
	\centering
	\includegraphics[width=.8\linewidth]{FFT-S3-W2}
	\caption{Wykresy w formie W2 FFT dla sygnału S3. Czas obliczania: 0,136300s}
	\label{S3_sygnal}
\end{figure}


\subsection{Transformacja Falkowa}
W tej sekcji przedstawiony zostanie wykres uzyskany z punktów otrzymanych po przeprowadzeniu transformacji falkowej na sygnale S1 dla 64 próbek.
\begin{figure}[H]
	\centering
	\includegraphics[width=\linewidth]{falkowa-s1}
	\caption{Wykres sygnału S1 po przeprowadzaniu na nim transformacji falkowej}
	\label{s1-falkowa}
\end{figure}
Czas transformacji 0.0001s.

%%%%%%%%%%%%%%%%%%%%%%%%%%%%%%%%%%%%%%%%%%%%%%%%%%%%%%%%%%%%%%%%%%%%%%%%%%%%%%%%%%%%%%%%%%%%%%%%%%%%%%%%%%%%%%%%%

%%%%%%%%%%%%%%%%%%%%%%%%%%%%%%%%%%%%%%%%%%%%%%%%%%%%%%%%%%%%%%%%%%%%%%%%%%%%%%%%%%%%%%%%%%%%%%%%%%%%%%%%%%%%%%%%%
\section{Wnioski}
Wyniki otrzymane dla wszystkich eksperymentów wydaja się prawidłowe. Dla transformacji DFT oraz FFT można zauważyć wyodrębnienie każdej ze składowych sygnału. Zaobserwować można również różnice w czasie przetwarzania między transformacją DFT oraz FFT, która w przybliżeniu wynosi dziesięciokrotność na korzyść transformacji FFT. Ponadto trzeba wspomnieć, że czas obliczeń może zmieniać się w zależności od chwilowego obciążenia procesora komputera oraz jego specyfikacji jak i konkretnych szczegółów implementacyjnych, które nie są charakterystyczne dla algorytmu ale dla danej jego implementacji.
Wyniki transformacji falkowej wyglądają zupełnie inaczej niż wyniki innych transformat co spowodowane jest tym, iż ta transformata przenosi funkcje w dziedzinę czasu. Można również zauważyć, że ta transformacja jest wręcz nieporównywalnie szybsza od DFT a nawet FFT.
%%%%%%%%%%%%%%%%%%%%%%%%%%%%%%%%%%%%%%%%%%%%%%%%%%%%%%%%%%%%%%%%%%%%%%%%%%%%%%%%%%%%%%%%%%%%%%%%%%%%%%%%%%%%%%%%%
% BIBLIOGRAFIA
%%%%%%%%%%%%%%%%%%%%%%%%%%%%%%%%%%%%%%%%%%%%%%%%%%%%%%%%%%%%%%%%%%%%%%%%%%%%%%%%%%%%%%%%%%%%%%%%%%%%%%%%%%%%%%%%%
\begin{thebibliography}{}
\bibitem{instrukcja} Instrukcja do zadania 4 na stronie przedmiotu. [przeglądany 26.05.2021], Dostępny w: https://ftims.edu.p.lodz.pl/pluginfile.php/14303/mod\_resource/content/0/zadanie4.pdf


\end{thebibliography}



\end{document}
