\documentclass[12pt]{article}
\usepackage[T1]{fontenc}
\usepackage[T1]{polski}
\usepackage[utf8]{inputenc}
\newcommand{\BibTeX}{{\sc Bib}\TeX} 
\usepackage{graphicx}
\usepackage{amsfonts}
\usepackage{amsmath}
\usepackage{latexsym}
\usepackage{float}

\setlength{\textheight}{21cm}

\title{{\bf Zadanie nr 1 - Generacja sygnału i szumu}\linebreak
Cyfrowe Przetwarzanie Sygnałów}
\author{Dawid Jakubik, 224307 \and Hubert Gawłowski, 224298}
\date{17.06.2021}

\begin{document}
\clearpage\maketitle
\thispagestyle{empty}
\newpage
\setcounter{page}{1}
\section{Cel zadania}

Celem zadania było zapoznanie się z operacjami transformacji sygnałów dyskretnych przy użyciu wybranych metod oraz zaimplementowanie tychże metod w celu wygenerowania odpowiednich transformat.

\section{Wstęp teoretyczny}
Podczas pracy nad zadaniem korzystaliśmy z teorii zawartej w instrukcji na platformie Wikamp \cite{instrukcja}. Znajdują się w niej wszystkie potrzebne wzory dotyczące interesujących nas transformacji.
Oczywiście, zgodnie z instrukcją zapewniliśmy dwa tryby wyświetlania wykresu:
\begin{itemize}
    \item (W1) - górny wykres prezentuje część rzeczywistą amplitudy w funkcji częstotliwości, a wykres dolny część urojoną.
    \item (W2) - górny wykres prezentuje moduł liczby zespolonej, a dolny argument liczby w funkcji częstotliwości.
\end{itemize}
W ramach zadania zaimplementowaliśmy wymienione w instrukcji transformacje sygnałów. Z racji numerów indeksów przydzielony nam został Zestaw 2, a więc wariant transformacji falkowej (DB4)


\section{Eksperymenty i wyniki}

%%%%%%%%%%%%%%%%%%%%%%%%%%%%%%%%%%%%%%%%%%%%%%%%%%%%%%%%%%%%%%%%%%%%%%%%%%%%%%%%%%%%%%%%%%%%%%%%%%%%%%%%%%%%%%%%%
% PODROZDZIA£ PT. EKSPERYMENT NR 1 
%%%%%%%%%%%%%%%%%%%%%%%%%%%%%%%%%%%%%%%%%%%%%%%%%%%%%%%%%%%%%%%%%%%%%%%%%%%%%%%%%%%%%%%%%%%%%%%%%%%%%%%%%%%%%%%%%

\subsection{Wykresy wejściowe}

Na początek wygenerowaliśmy wykresy o następujących wzorach, na których będą przeprowadzane transformacje:
\begin{itemize}
    \item (S1):
    \begin{equation}
        S(t) = 2 \sin (\frac{2 \pi}{2}t + \frac{\pi}{2}) + 5 \sin (\frac{2 \pi}{0.5}t + \frac{\pi}{2})f_{pr} = 16
    \end{equation}
    \item (S2):
    \begin{equation}
        S(t) = 2 \sin (\frac{2 \pi}{2}t) + \sin (\frac{2 \pi}{1}t)+5 \sin (\frac{2\pi}{0.5}t) f_{pr} = 16
    \end{equation}
    \item (S3):
    \begin{equation}
        S(t) = 5 \sin (\frac{2 \pi}{2}t) + \sin (\frac{2 \pi}{0.25}t)f_{pr} = 16
    \end{equation}
\end{itemize}


%%%%%%%%%%%%%%%%%%%%%%%%%%%%%%%%%%%%%%%%%%%%%%%%%%%%%%%%%%%%%%%%%%%%%%%%%%%%%%%%%%%%%%%%%%%%%%%%%%%%%%%%%%%%%%%%%

%%%%%%%%%%%%%%%%%%%%%%%%%%%%%%%%%%%%%%%%%%%%%%%%%%%%%%%%%%%%%%%%%%%%%%%%%%%%%%%%%%%%%%%%%%%%%%%%%%%%%%%%%%%%%%%%%
\section{Wnioski}

%%%%%%%%%%%%%%%%%%%%%%%%%%%%%%%%%%%%%%%%%%%%%%%%%%%%%%%%%%%%%%%%%%%%%%%%%%%%%%%%%%%%%%%%%%%%%%%%%%%%%%%%%%%%%%%%%
% BIBLIOGRAFIA
%%%%%%%%%%%%%%%%%%%%%%%%%%%%%%%%%%%%%%%%%%%%%%%%%%%%%%%%%%%%%%%%%%%%%%%%%%%%%%%%%%%%%%%%%%%%%%%%%%%%%%%%%%%%%%%%%
\begin{thebibliography}{}
\bibitem{instrukcja} Instrukcja do zadania 4 na stronie przedmiotu. [przeglądany 26.05.2021], Dostępny w: https://ftims.edu.p.lodz.pl/pluginfile.php/14303/mod\_resource/content/0/zadanie4.pdf


\end{thebibliography}



\end{document}
